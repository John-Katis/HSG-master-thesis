% ******************************************************************************
% ****************************** Custom Margin *********************************

% Add `custommargin' in the document class options to use this section
% Set {innerside margin / outerside margin / topmargin / bottom margin}  and
% other page dimensions
\ifsetCustomMargin
  \RequirePackage[left=25mm,right=25mm,top=29mm,bottom=25mm]{geometry}
  %\RequirePackage[left=37mm,right=30mm,top=35mm,bottom=30mm]{geometry}
  \setFancyHdr % To apply fancy header after geometry package is loaded
\fi

% Add spaces between paragraphs
%\setlength{\parskip}{0.5em}
% Ragged bottom avoids extra whitespaces between paragraphs
% \raggedbottom
% To remove the excess top spacing for enumeration, list and description
%\usepackage{enumitem}
%\setlist[enumerate,itemize,description]{topsep=0em}

% *****************************************************************************
% ******************* Fonts (like different typewriter fonts etc.)*************

% Add `customfont' in the document class option to use this section

%\ifsetCustomFont
  % Set your custom font here and use `customfont' in options. Leave empty to
  % load computer modern font (default LaTeX font).
  %\RequirePackage{helvet}

  % For use with XeLaTeX
  %  \setmainfont[
  %    Path              = ./libertine/opentype/,
  %    Extension         = .otf,
  %    UprightFont = LinLibertine_R,
  %    BoldFont = LinLibertine_RZ, % Linux Libertine O Regular Semibold
  %    ItalicFont = LinLibertine_RI,
  %    BoldItalicFont = LinLibertine_RZI, % Linux Libertine O Regular Semibold Italic
  %  ]
  %  {libertine}
  %  % load font from system font
  %  \newfontfamily\libertinesystemfont{Linux Libertine O}
%\fi

% *****************************************************************************
% **************************** Custom Packages ********************************

% ************************* My Custom Packages *********************************
\usepackage[a-2b]{pdfx}
\usepackage[export]{adjustbox}
\usepackage{color, colortbl}
\usepackage{enumitem}
\usepackage{tcolorbox}
\usepackage{setspace}
\usepackage[nopostdot,nonumberlist,acronym,toc]{glossaries}
\usepackage{xurl}
\usepackage[fontsize=13]{fontsize}
\usepackage{caption}
\usepackage{listings}
\usepackage{mathpartir}
\usepackage{newfloat}
\usepackage{overpic}
\usepackage{rotating}
\usepackage{tikz}
\usetikzlibrary{shapes,arrows.meta,positioning,fit,calc,tikzmark}
\usepackage{tipa}
\captionsetup[table]{font={stretch=1.3},labelfont=bf}     %% change 1.2 as you like
\captionsetup[figure]{font={stretch=1.3}, labelfont=bf}
\usepackage[normalem]{ulem}
\newcommand{\stkout}[1]{\ifmmode\text{\sout{\ensuremath{#1}}}\else\sout{#1}\fi}

% ************************* Algorithms and Pseudocode **************************

\usepackage{algorithm}
\usepackage[noend]{algpseudocode}

% knuth style line numbers (3.5 instead of 5)
\algrenewcommand{\alglinenumber}[1]{\makebox[2.5em]{\thealgorithm.#1}}
\makeatletter 
\renewcommand\thealgorithm{\thechapter.\arabic{algorithm}} 
\@addtoreset{algorithm}{chapter} 
\makeatother


% ******************************** Math Stuff **********************************

\usepackage{amsthm}
\theoremstyle{definition}
\newtheorem{definition}{Definition}[chapter]
\newtheorem{theorem}{Theorem}[chapter]

% ********************Captions and Hyperreferencing / URL **********************

% Captions: This makes captions of figures use a boldfaced small font.
%\RequirePackage[small,bf]{caption}


% *************************** Graphics and figures *****************************

%\usepackage{rotating}
%\usepackage{wrapfig}

% Uncomment the following two lines to force Latex to place the figure.
% Use [H] when including graphics. Note 'H' instead of 'h'
%\usepackage{float}
%\restylefloat{figure}

% Subcaption package is also available in the sty folder you can use that by
% uncommenting the following line
% This is for people stuck with older versions of texlive
%\usepackage{sty/caption/subcaption}
\usepackage{subcaption}

% ********************************** Tables ************************************
\usepackage{booktabs} % For professional looking tables
\usepackage{multirow}
\usepackage{multicol}
\usepackage{lipsum} 
%\usepackage{longtable}
%\usepackage{tabularx}


% *********************************** SI Units *********************************
\usepackage{siunitx} % use this package module for SI units


% ******************************* Line Spacing *********************************

% Choose linespacing as appropriate. Default is one-half line spacing as per the
% University guidelines

% \doublespacing
% \onehalfspacing
% \singlespacing
\setstretch{1.3} % 1.4 or 1.3 as per HSG printing guidelines


% ************************ Formatting / Footnote *******************************

% Don't break enumeration (etc.) across pages in an ugly manner (default 10000)
%\clubpenalty=500
%\widowpenalty=500

%\usepackage[perpage]{footmisc} %Range of footnote options


% *****************************************************************************
% *************************** Bibliography  and References ********************

\usepackage{cleveref} %Referencing without need to explicitly state fig /table


\makeatletter
% Named labels: First argument is reference, second is display text
\def\namedlabel#1#2{\begingroup
	#2%
	\def\@currentlabel{#2}%
	\phantomsection\label{#1}\endgroup
}
\def\hiddennamedlabel#1#2{\begingroup
	\def\@currentlabel{#2}%
	\phantomsection\label{#1}\endgroup
}
\def\namedseclabel#1#2{\begingroup
	#2%
	\def\@currentlabel{#2}%
	\label{#1}\endgroup
}
\makeatother

%\RequirePackage[backend=biber, style=numeric-comp, citestyle=numeric, sorting=nty, natbib=true]{biblatex}
%\bibliography{References/references} %Location of references.bib only for biblatex
\RequirePackage[backend=biber, style=numeric-comp, citestyle=numeric, sorting=nty, natbib=true, maxbibnames=99, maxcitenames=2, sortcites]{biblatex}
\bibliography{
  bibliography/bibliography.bib
} %Location of references.bib only for biblatex

\newcommand{\bibIgnore}{%
  \ifentrytype{misc}{}{%
    \clearfield{urlyear}%
    \clearfield{urlmonth}%
    \clearfield{urldate}%
    \clearfield{note}%
    \clearfield{url}%
    \clearfield{month}%
    \clearfield{series}%
    \clearfield{eprint}%
    \clearname{editor}%
    \clearlist{language}%
    \clearlist{publisher}%
    \clearlist{location}%
  }%
}
\AtEveryBibitem{\bibIgnore}
\AtEveryCitekey{\bibIgnore}

% Reference appendices as Appendix, not Chapter, and change chapter title format
\let\oldappendices\appendices
\let\endoldappendices\endappendices
\renewenvironment{appendices}{%
  \oldappendices
  \crefalias{chapter}{appendix}
  \titleformat{\chapter}[display]
    {\normalfont\Large\bfseries}
    {\chaptertitlename\ \thechapter}{0pt}{\Large\bfseries}
}{\endoldappendices}

% Dirty hack needed to show more than 2 names in the publications overview
\makeatletter
\newrobustcmd{\getmaxcitenames}{\blx@maxcitenames}
\newrobustcmd*{\setmaxcitenames}{\numdef\blx@maxcitenames}
\makeatother

% *****************************************************************************
% *************************** Glossary and Acronyms ***************************

%\makeglossaries
%\newacronym{ATCF}{ATCF}{As-The-Crow-Flies}
%\newacronym{LAS}{LAS}{Least Angle Strategy}
%\newacronym{TBT}{TBT}{Turn-By-Turn}

%\glsadd{ATCF}\glsadd{TBT}\glsadd{LAS}


% ********************** TOC depth and numbering depth *************************

\setcounter{secnumdepth}{2}
\setcounter{tocdepth}{2}

% ******************************* Nomenclature *********************************

% To change the name of the Nomenclature section, uncomment the following line

%\renewcommand{\nomname}{Symbols}


% ********************************* Appendix ***********************************

% The default value of both \appendixtocname and \appendixpagename is `Appendices'. These names can all be changed via:

\renewcommand{\appendixtocname}{List of appendices}
\renewcommand{\appendixname}{Appendix}

% *********************** Configure Draft Mode **********************************

% Uncomment to disable figures in `draft'
%\setkeys{Gin}{draft=true}  % set draft to false to enable figures in `draft'

% These options are active only during the draft mode
% Default text is "Draft"
%\SetDraftText{DRAFT}

% Default Watermark location is top. Location (top/bottom)
%\SetDraftWMPosition{bottom}

% Draft Version - default is v1.0
%\SetDraftVersion{v1.1}

% Draft Text grayscale value (should be between 0-black and 1-white)
% Default value is 0.75
%\SetDraftGrayScale{0.8}


% ******************************** Todo Notes **********************************
%% Uncomment the following lines to have todonotes.

%\ifsetDraft
%	\usepackage[colorinlistoftodos]{todonotes}
%	\newcommand{\mynote}[1]{\todo[author=kks32,size=\small,inline,color=green!40]{#1}}
%\else
%	\newcommand{\mynote}[1]{}
%	\newcommand{\listoftodos}{}
%\fi

% Example todo: \mynote{Hey! I have a note}


% ******************************** Listings **********************************

\DeclareFloatingEnvironment[
  fileext=lsts,
  name=Listing
]{listing}
\DeclareCaptionSubType{listing}

\crefformat{sublisting}{Sublisting #2#1#3}
\crefmultiformat{sublisting}{Sublisting #2#1#3}{ and~#2#1#3}{, #2#1#3}{, and~#2#1#3}
\crefrangeformat{sublisting}{Sublistings #3#1#4 to~#5#2#6}

% Bind listings and lstlistings counters
\usepackage{xassoccnt}
\DeclareCoupledCounters[name=listinglstlisting]{listing,lstlisting}

% knuth style line numbers (3.5 instead of 5)
\edef\thelstnumber{%
  \unexpanded\expandafter{\arabic{chapter}}.%
  \unexpanded\expandafter{\arabic{lstlisting}}%
  \unexpanded\expandafter{\alph{sublisting}}.%
  \unexpanded\expandafter{\thelstnumber}%
}
\makeatletter
\let\orig@lstnumber=\thelstnumber
\newcommand\lstsetnumber[1]{\gdef\thelstnumber{#1}}
\newcommand\lstresetnumber{\global\let\thelstnumber=\orig@lstnumber}
\makeatother

\lstdefinelanguage{JavaScript}{
	morekeywords=[1]{break, continue, delete, else, for, function, if, in,
		new, return, this, typeof, var, while}, %, with},
	% Literals, primitive types, and reference types.
	morekeywords=[2]{false, null, true, boolean, number, string, void, undefined,
		Array, Boolean, Date, Math, Number, String, Object},
	% Built-ins.
	morekeywords=[3]{eval, parseInt, parseFloat, escape, unescape},
	sensitive,
	morecomment=[s]{/*\ }{*/},
	morecomment=[l]//,
	morecomment=[s]{/**\ }{*/}, % JavaDoc style comments
	morestring=[b]',
	morestring=[b]"
}[keywords, comments, strings]

\lstalias[]{ES6}[ECMAScript2015]{JavaScript}
\lstdefinelanguage[ECMAScript2015]{JavaScript}[]{JavaScript}{
	morekeywords=[1]{await, async, as, case, catch, class, const, default, do,
		enum, export, extends, finally, from, implements, import, instanceof, interface,
		let, static, super, switch, throw, try},
	morestring=*[b]`, % Interpolation strings.
    morestring=[s][]{\$\{}{\}} % Expressions in interpolation strings.
}

\definecolor{mediumgray}{rgb}{0.3, 0.4, 0.4}
\definecolor{mediumblue}{rgb}{0.0, 0.0, 0.8}
\definecolor{forestgreen}{rgb}{0.13, 0.55, 0.13}
\definecolor{darkviolet}{rgb}{0.58, 0.0, 0.83}
\definecolor{royalblue}{rgb}{0.25, 0.41, 0.88}
\definecolor{crimson}{rgb}{0.86, 0.8, 0.24}

\newcommand{\lstcomment}{\color{mediumgray}\upshape\rmfamily}
\lstdefinestyle{JSES6Base}{
	aboveskip=0pt,
	belowskip=0pt,
	moredelim=**[is][\color{orange}]{@@}{@@}, % ProTI spec highlighting
	backgroundcolor=\color{white},
	basicstyle=\ttfamily\lst@ifdisplaystyle\footnotesize\fi,
	breakatwhitespace=false,
	breaklines=true,
	columns=fullflexible,
	commentstyle=\lstcomment,
	emph={},
	emphstyle=\color{crimson},
	extendedchars=true,  % requires inputenc
	fontadjust=true,
	frame=tb,
	identifierstyle=\color{black},
	keepspaces=true,
	keywordstyle=\color{mediumblue},
	keywordstyle={[2]\color{darkviolet}},
	keywordstyle={[3]\color{royalblue}},
	numbers=left,
	numbersep=5pt,
	numberstyle=\color{black}\footnotesize,
	rulecolor=\color{black},
	showlines=true,
	showspaces=false,
	showstringspaces=false,
	showtabs=false,
	stringstyle=\color{forestgreen},
	tabsize=2,
	title=\lstname,
	xleftmargin=3em,
	framexleftmargin=3em,
	upquote=true  % requires textcomp
}

\lstdefinestyle{JavaScript}{
	language=JavaScript,
	style=JSES6Base
}
\lstdefinestyle{ES6}{
	language=ES6,
	style=JSES6Base
}

\lstset{
	style=ES6,
	escapechar={~}
}


% ******************************** Inference Rules **********************************

\makeatletter
\let\originferrule\inferrule%
\DeclareDocumentCommand\inferrule{ s O {} m m o }{%
  \IfBooleanTF{#1}%
  {%
    \mpr@inferstar[#2]{#3}{#4}%
  }{%
    \mpr@inferrule[#2]{#3}{#4}%
  }%
  \IfValueT{#5}%
  {%
    \quad
    \textsc{#5}%
    \my@name@inferrule{\textsc{#5}}%
  }%
}
\NewDocumentCommand\my@name@inferrule{ m }{%
  \def\@currentlabelname{\ensuremath{#1}}%
}
\makeatother



% ******************************** Daniel's Diss Macros **********************************

\newcommand{\TODO}[1]{{\color{red} #1}}

\newcommand{\RQ}[2]{\paragraph[\ref{#1:RQ#2}]{\namedlabel{#1:RQ#2}{RQ\,\thechapter.#2}}}
\newcommand{\RI}[2]{\paragraph[\ref{#1:RI#2}]{\namedlabel{#1:RI#2}{RI\,\thechapter.#2}}}
